
\documentclass[11pt]{article}

%%% PAQUETES

% Idioma
\usepackage[utf8]{inputenc}
\usepackage[spanish]{babel}
\setlength{\parindent}{0pt}

% Matemáticas
\usepackage{amssymb, amsthm}

% Fuentes
\usepackage[T1]{fontenc}
\usepackage[sfdefault, scaled=.85]{roboto}
\usepackage[scaled=.8]{FiraMono}
\usepackage[cmintegrals]{newtxsf}
\usepackage[italic]{mathastext}
\usepackage{textcomp}
\usepackage{wasysym}

% Tablas
\usepackage{multirow}
\usepackage{colortbl}

% Gráficos y colores

\usepackage[x11names, rgb, html]{xcolor}
\usepackage{graphics}
\usepackage{caption}
\usepackage{float}
\usepackage{adjustbox}

% Ajustes del documento
\usepackage{geometry}
\geometry{left=3cm,right=3cm,top=3cm,bottom=3cm,headheight=1cm,headsep=0.5cm}
\usepackage{enumitem}

% Entornos personalizados.
\usepackage{mdframed}

% Código
\usepackage{listingsutf8}

%%% COLORES

% Material Design

\definecolor{50}{HTML}{E0F7FA}
\definecolor{300}{HTML}{4DD0E1}
\definecolor{500}{HTML}{00BCD4}
\definecolor{700}{HTML}{0097A7}
\definecolor{900}{HTML}{006064}

% Solarized

\definecolor{sbase03}{HTML}{002B36}
\definecolor{sbase02}{HTML}{073642}
\definecolor{sbase01}{HTML}{586E75}
\definecolor{sbase00}{HTML}{657B83}
\definecolor{sbase0}{HTML}{839496}
\definecolor{sbase1}{HTML}{93A1A1}
\definecolor{sbase2}{HTML}{EEE8D5}
\definecolor{sbase3}{HTML}{FDF6E3}
\definecolor{syellow}{HTML}{B58900}
\definecolor{sorange}{HTML}{CB4B16}
\definecolor{sred}{HTML}{DC322F}
\definecolor{smagenta}{HTML}{D33682}
\definecolor{sviolet}{HTML}{6C71C4}
\definecolor{sblue}{HTML}{268BD2}
\definecolor{scyan}{HTML}{2AA198}
\definecolor{sgreen}{HTML}{859900}

% Colores del documento

\definecolor{text}{RGB}{78,78,78}
\definecolor{accent}{RGB}{129, 26, 24}

%%% LISTINGS

% Listing -> Código fuente
\renewcommand{\lstlistingname}{Código fuente}

% Ajustes para que funcionen bien las tildes de los comentarios

\lstset{
  inputencoding=utf8/latin1
}

% Ajustes de Listings para el documento

\lstset{
  frame=leftline,
  rulecolor=\color{300},
  framerule=2pt,
  % Números de línea
  numbers=left,
  % Margen adicional para alinear los entornos con el resto de párrafos
  xleftmargin=0.7em,
  % Espacio adicional debajo del título
  belowcaptionskip=1\baselineskip,
  % Colores
  basicstyle=\footnotesize\ttfamily\color{sbase00},
  keywordstyle=\color{700},
  commentstyle=\color{300},
  stringstyle=\color{500},
  numberstyle=\color{500},
  % Separar líenas largas en varias líneas
  breaklines=true,
  showstringspaces=false,
  tabsize=2,
}

%%% ENTORNOS PERSONALIZADOS

\newtheoremstyle{ejercicio-style} % Nombre del estilo.
{-0.5em}                          % Espacio por encima.
{}                                % Espacio por debajo.
{\normalfont}                     % Fuente del cuerpo.
{}                                % Identación.
{\bf\sffamily}                    % Fuente para la cabecera.
{.}                               % Puntuación tras la cabecera.
{.5em}                            % Espacio tras la cabecera.
{\thmname{#1}\thmnumber{ #2}\thmnote{ (#3)}}     % Especificación de la cabecera (actual: nombre en negrita).


\mdfdefinestyle{ejercicio-frame}{
  linewidth=2pt, %
  linecolor= 300, %
  backgroundcolor= 50,
  topline=false, %
  bottomline=false, %
  rightline=false,%
  leftmargin=0pt, %
  innerleftmargin=1em, %
  innerrightmargin=1em,
  rightmargin=0pt, %
  innertopmargin=1em,%
  innerbottommargin=1em, %
  splittopskip=\topskip, %
}%

\surroundwithmdframed[style=ejercicio-frame]{ejer}

\theoremstyle{ejercicio-style}
\newtheorem{ejer}{Ejercicio}

\usepackage{tikz}
\usetikzlibrary{automata,positioning}

\begin{document}

% Cabecera del documento

\begin{tabular*}{\textwidth}{@{\extracolsep{\fill}}!{\color{300}{\vrule width 2pt}}>{\columncolor{50}}clc}
    \noalign{\global\arrayrulewidth=2pt}
    %\arrayrulecolor{300}\hline
    & & \\
    & \Large{Modelos de Computación (MC)} & \\
               & \large{Entrega de ejercicios.} & \\
               & \large{Tema 6} & \\
          & & \\
          & \textsf{Estudiante: Miguel Lentisco Ballesteros}  & \\
          & \textsf{Grupo de prácticas: A1} & \\
          & \textsf{Fecha de entrega: 28/12/2017} & \\
         \multirow{-10}{*}{ \begin{tabular}{c}
        \small{3º curso / 1º cuatr.} \\ Grado Ing. Inform. \\ Doble Grado Ing. \\ Inform. y Mat.
\end{tabular}}  & & \\
    %\hline
\end{tabular*}

\vspace{1cm}

\section*{Ejercicios Tema 6}
\label{sec:ej_tema_6}

\begin{ejer}
Dar gramáticas independientes del contexto que generen los siguientes lenguajes sobre el alfabeto $A = \{0,1\}$:
	\begin{enumerate}
		\item $L_1$: conjunto de palabras tal que si la palabra empieza por $0$, entonces tiene el mismo número de $0$s que de $1$s.
		\item $L_2$: conjunto de palabras tal que si la palabra termina por $1$, entonces tiene un número de $1$s mayor o igual que el número de $0$s.
		\item $L_2\cap L_2$
	\end{enumerate}
\end{ejer}

\emph{Respuesta.}
\begin{enumerate}
	\item $L_1$ está generado por la gramática con estas producciones:
	$$ S \rightarrow 1X \ |\  0Y1Y \ | \ \epsilon $$
	$$ X \rightarrow 1X \ |\ OX \ | \ \epsilon $$
	$$ Y \rightarrow 0Y1Y \ | \ 1Y0Y \ | \ \epsilon $$
	\item $L_2$ está generado por la gramática con estas producciones:
	$$ S \rightarrow X0 \ |\  Y1 \ | \ \epsilon $$
	$$ X \rightarrow X1 \ |\ X0 \ | \ \epsilon $$
	$$ Y \rightarrow Y1 \ | \ Y0Y1 \ | \ Y1Y0 \ | \ \epsilon $$
	\item La intersección sería entonces:
	$$ S \rightarrow 0Y1Y \ |\  1Z1 \ | \ 1X0 \ | \ 1 \ | \ \epsilon $$
	$$ X \rightarrow 1X \ |\ OX \ | \ \epsilon $$
	$$ Y \rightarrow 0Y1Y \ | \ 1Y0Y \ | \ \epsilon $$
	$$ Y \rightarrow 1Z \ | \ 1Z0Z \ | \ 0Z1Z \ | \ \epsilon $$
\end{enumerate}
	
\begin{ejer}
Una gramática independiente del contexto generalizada es una gramática en el que las producciones son de la forma $A \rightarrow \emph{r}$ donde \emph{r} es una expresión regular de variables y símbolos terminales. Una gramática independiente del contexto generalizada representa una forma compacta de representar una gramática con todas las producciones $A \rightarrow \alpha$ donde $\alpha$ es una palabra del lenguaje asociado a la expresión regular \emph{r} y $A \rightarrow \emph{r}$ es una producción de la gramática generalizada. Observemos que esta gramática asociada puede tener infinitas producciones, ya que una expresión regular puede representar un lenguaje con infinitas palabras. El concepto de lenguaje generado por una gramática generalizada se define de forma análoga al de las gramáticas independientes del contexto, pero teniendo en cuenta que ahora puede haber infinitas producciones. Demostrar que un lenguaje es independiente del contexto si y solo si se puede generar por una gramática generalizada.
\end{ejer}

\emph{Respuesta.} Tenemos que demostrar lo siguiente:
	\begin{center}
		Lenguaje $L$ independiente del contexto $\iff$ $L=L(G)$, G es una gramática generalizada
	\end{center}
	\framebox{$\Rightarrow$} Esto es trivial, ya que si el lenguaje $L$ es independiente del contexto entonces existe una gramática independiente del contexto $G$ que la genera, viendo facilmente que $A \rightarrow r$ como r es una expresión regular de variables y símbolos terminales, se puede ver en concreto como expresiones regulares, luego es una gramática generalizada. \\
	
\framebox{$\Leftarrow$} Tenemos que ver que con una gramática generalizada podemos construir una gramática independiente del contexto, así que vamos a ver como. Sabemos que tenemos expresiónes regulares, asi que la idea es que si tenemos una expresión regular en general poder convertirla en expresiones más simples, por ejemplo:

$$ A \rightarrow R^* \equiv A \rightarrow RR \ | \ \epsilon $$
$$ A \rightarrow R_1 + R_2 \equiv A \rightarrow R_1 \ | \ R_2 $$
$$ A \rightarrow R_1R_2 \equiv A \rightarrow R_1 + \cdots + R_k $$ esto último si es posible descomponiendo aplicando la distribución de la suma, si no es posible se deja igual, como concatenación, por ejemplo:
$$ A \rightarrow CD \equiv A \rightarrow CD $$
$$ A \rightarrow (X+Y)A \equiv A \rightarrow XA+YA \equiv A \rightarrow XA \ | \ XY $$
Así una concatenación con $*$, usando lo de arriba sería:
$$ A \rightarrow R_1R_2^*R_3 \equiv A \rightarrow R_1R_2R_3, R_2 \rightarrow R_2R_2 \ | \ \epsilon $$
Por lo que en este caso ampliaríamos las producciones. \\


Aplicando todas estas reglas podemos transformar cualquier expresión regular compleja en simples, teniendo así las producciones simples, y por tanto una gramática independiente del contexto y por ende, el lenguaje generado es independiente del contexto.

\newpage

\begin{ejer}
Demostrar que los siguientes lenguajes son independientes del contexto:
\begin{enumerate}
		\item $L_1 = \{ \ u\#v: u^{-1} \text{ es una subcadena de } w, u,w \in \{0,1\}^* \} $
	\item $L_2 = \{ \ u_1\# u_2\# \cdots \#u_k: k \geq 1 , \text{ cada } u_i\in\{0,1\}^*, \text{ y  para algún } i \ y \ j, u_i = u_j^{-1} \} $
\end{enumerate}
\end{ejer}

\emph{Respuesta.}
\begin{enumerate}
	\item Es facil ver que tiene que ser de la forma $u\#yu^{-1}z$, luego usando estas producciones:
	
	$$ S \rightarrow YX \ $$
	$$ X \rightarrow 1X \ | \ 0X \ | \ \epsilon $$
	$$ Y \rightarrow 1Y1 \ | \ 0Y0 \ | \ \#X $$
	
	\item La idea es que hagamos un palídromo primero, el $u_i$ y el $u_j$ permitiendo lo que sea a los lados y luego pudiendo rellenar entre medias, también teniendo en cuenta el caso especial para $k=1$ que sería solo palíndromos, entonces:
	
	$$ S \rightarrow AMB $$
	$$ A \rightarrow AA \ | \ X\# \ | \ \epsilon $$
	$$ B \rightarrow BB \ | \ \#X \ | \ \epsilon $$
	$$ X \rightarrow 1X \ | \ 0X \ | \ \epsilon $$
	$$ M \rightarrow 1M1 \ | \ 0M0 \ | \ \#A \ | \ \epsilon $$
\end{enumerate}

\newpage
\begin{ejer}
Sea el lenguaje $L = \{ \ u\#v : u,v \in \{0,1\}^*, u \neq v \} $, demostrar que es independiente del contexto.
\end{ejer}

\emph{Respuesta.} Buscaremos una gramática que genere este lenguaje, para ello tenemos que fijarnos en dos posibilidades, si el modulo de u es distinto de v; o bien que el módulo sea igual pero entonces sean distintas en sí las palabras.

\begin{itemize}
\item Primero la gramática para las palabras de distinto módulo:

$$ S \rightarrow \#Y \ | \ Y\# \ | \ XSX $$
$$ X \rightarrow 0 \ | \ 1 $$
$$ Y \rightarrow X \ | \ XY \ $$

\item Luego para las que son de mismo módulo pero diferentes no he llegado a encontrar ninguna gramática, pero la idea sería encontrarla y con las unión de ambas tenemos una gramática independiente del contexto que genera el lenguaje deseado.
\end{itemize}


\newpage

\begin{ejer}
	Dar gramáticas independientes del contexto no ambiguas para los siguientes lenguajes sobre el alfabeto $\{0,1\}^*$:
	\begin{enumerate}
		\item El conjunto de palabras $w$ tal que en todo prefijo de $w$ el número de $0$s es mayor o igual que el número de $1$s.
		\item El conjunto de palabras $w$ en las que el número de $0$s es mayor o igual que el número de $1$s.
	\end{enumerate}
\end{ejer}

\emph{Respuesta.} Sé que las gramáticas son ambiguas pero no consigo encontrar unas que sean no ambiguas:
\begin{enumerate}
	\item La gramática sería:
	
	$$ S \rightarrow 0X \ | \ \epsilon $$
	$$ X \rightarrow 0X \ | \ 1X0X \ | \ 0X1X \ | \ \epsilon $$
	
	\item La gramática sería:
	
	$$ S \rightarrow 0S \ | \ 1S0S \ | \ 0S1S \ | \ \epsilon $$
\end{enumerate}
\end{document}